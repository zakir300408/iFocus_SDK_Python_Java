\documentclass[11pt,a4paper]{article}

\usepackage[margin=2.5cm]{geometry}
\usepackage{enumitem}
\usepackage{hyperref}
\usepackage{amssymb}   % for \square
\usepackage{graphicx}  % for logo
\usepackage{fancyhdr}  % for header/footer

\setlength{\headheight}{20pt}
\pagestyle{fancy}
\fancyhf{}
\lhead{\includegraphics[height=1cm]{logo.png}}
\rhead{Niantong Intelligence}
\cfoot{\thepage}

\begin{document}
	
	\begin{center}
		{\LARGE \textbf{iFocus Android SDK -- Deliverables \& Function Specification}}\\[0.5em]
		{\large Niantong Intelligence}\\[1em]
	\end{center}
	
	\section*{Overview}
	
	This document describes the deliverables and function-level specification for the iFocus Android SDK.
	
	\begin{itemize}[leftmargin=1.5em]
		\item The SDK connects to the iFocus EEG device over Bluetooth on an Android tablet.
		\item The SDK collects EEG signals internally and outputs a per-subject focus/attention score in real time.
		\item All signal processing and model logic are internal; the client application only interacts with high-level functions and callbacks.
	\end{itemize}
	
	\section*{Implementation Language \& Packaging}
	
	\begin{itemize}[leftmargin=1.5em]
		\item The SDK will be implemented in \textbf{Kotlin} for Android.
		\item All public APIs will be usable from both \textbf{Kotlin and Java} (full Java interoperability).
		\item The SDK will be distributed as an Android \texttt{.aar} file or via a Maven dependency.
		\item All processing is performed locally on-device; no network connection is required (offline support).
	\end{itemize}
	
	\section*{Task 1: Device Integration}
	
	\subsection*{1.1 Bluetooth Scanning \& Connection}
	
	\paragraph{Scan for devices (will be provided).}
	
	\begin{itemize}[leftmargin=1.5em]
		\item \textbf{Function:} \texttt{startDeviceScan(callback)}
		\begin{itemize}
			\item Starts Bluetooth scanning for nearby iFocus devices.
			\item The callback delivers, for each discovered device:
			\begin{itemize}
				\item Device name.
				\item MAC address / deviceId.
				\item RSSI at discovery time.
			\end{itemize}
		\end{itemize}
		
		\item \textbf{Function:} \texttt{stopDeviceScan()}
		\begin{itemize}
			\item Stops scanning for nearby devices.
		\end{itemize}
	\end{itemize}
	
	\paragraph{Connect / disconnect (will be provided).}
	
	\begin{itemize}[leftmargin=1.5em]
		\item \textbf{Function:} \texttt{connect(deviceId, autoReconnect = false, callback)}
		\begin{itemize}
			\item Establishes a BLE connection to the specified device.
			\item If \texttt{autoReconnect} is \texttt{true}, the SDK will automatically attempt to reconnect if the BLE link drops unexpectedly.
			\item The callback provides:
			\begin{itemize}
				\item A success flag (true/false).
				\item An error code if the connection fails.
			\end{itemize}
		\end{itemize}
		
		\item \textbf{Function:} \texttt{disconnect()}
		\begin{itemize}
			\item Disconnects from the currently connected device.
			\item A disconnection event will be reported via the appropriate callback.
		\end{itemize}
	\end{itemize}
	
	\subsection*{1.2 RSSI (Bluetooth Signal Strength)}
	
	\paragraph{RSSI access (will be provided).}
	
	\begin{itemize}[leftmargin=1.5em]
		\item \textbf{Function:} \texttt{getCurrentRssi()}
		\begin{itemize}
			\item Returns the most recently known RSSI value for the active connection.
			\item The application may call this function whenever it wishes to read the current link quality.
		\end{itemize}
	\end{itemize}
	
	\paragraph{Note on signal quality.}
	
	\begin{itemize}[leftmargin=1.5em]
		\item The current device firmware does not expose a separate EEG contact/signal-quality metric.
		\item The SDK will provide \textbf{Bluetooth link quality} via RSSI only.
		\item Any ``quality display'' in the demo will be based on RSSI.
	\end{itemize}
	
	\subsection*{1.3 Battery Level}
	
	\paragraph{Battery level (not supported).}
	
	\begin{itemize}[leftmargin=1.5em]
		\item The current iFocus device firmware does not expose battery information.
		\item No battery-related API will be provided in the SDK.
	\end{itemize}
	
	\subsection*{1.4 Wearing Detection}
	
	\paragraph{Wearing status (will be provided).}
	
	\begin{itemize}[leftmargin=1.5em]
		\item \textbf{Function:} \texttt{getWearingStatus()}
		\begin{itemize}
			\item Returns the latest known wearing status:
			\begin{itemize}
				\item \texttt{true} = the device is detected as being worn.
				\item \texttt{false} = the device is detected as not being worn.
			\end{itemize}
			\item The application may call this function whenever it wishes to read the current wearing state.
		\end{itemize}
	\end{itemize}
	
	\section*{Task 2: Focus Model Pipeline (Per Subject)}
	
	For each subject (user), the SDK supports the following stages:
	
	\begin{enumerate}[leftmargin=1.5em]
		\item Calibration data collection for two mental states (focus, relax).
		\item Model training using the stored calibration data for that subject.
		\item Focus inference using the trained subject-specific model and live EEG data.
	\end{enumerate}
	
	All EEG data and model parameters are handled internally by the SDK; the client does not need to manage raw training data or models.
	
	\subsection*{2.1 Calibration Data Collection}
	
	Calibration is used to collect labeled data for two mental states (such as focus and relax). The application can start and stop calibration for each state, and repeat as needed.
	
	\paragraph{Start calibration (will be provided).}
	
	\begin{itemize}[leftmargin=1.5em]
		\item \textbf{Function:} \texttt{startCalibration(subjectId, stateLabel)}
		\begin{itemize}
			\item Deletes previously recorded calibration data for the specified subject.
			\item Allows the application to reset and perform a fresh calibration if desired.
			\item Begins recording calibration data for the specified subject.
			\item \texttt{stateLabel} indicates the mental state during this period (e.g., FOCUS, RELAX).
			\item While calibration is active, the SDK continuously records EEG data from the connected device and tags it with:
			\begin{itemize}
				\item \texttt{subjectId}.
				\item \texttt{stateLabel}.
			\end{itemize}
		\end{itemize}
	\end{itemize}
	
	\paragraph{Stop calibration (will be provided).}
	
	\begin{itemize}[leftmargin=1.5em]
		\item \textbf{Function:} \texttt{stopCalibration()}
		\begin{itemize}
			\item Stops recording calibration data for the current session.
			\item The collected data segment is stored internally for the active subject and state label.
			\item No model is trained at this step; this function only stops data collection.
		\end{itemize}
	\end{itemize}
	
	
	\subsection*{2.2 Train Focus Model}
	
	Once calibration data for different conditions (e.g., focus and relax) has been collected, the SDK can train a subject-specific focus model.
	
	\paragraph{Train and save model (will be provided).}
	
	\begin{itemize}[leftmargin=1.5em]
		\item \textbf{Function:} \texttt{trainFocusModel(subjectId, callback)}
		\begin{itemize}
			\item Loads all stored calibration data for the given subject.
			\item Trains a focus/attention model internally for that subject.
			\item Saves the trained model associated with the subject ID.
			\item The callback reports:
			\begin{itemize}
				\item A success flag (true/false).
				\item A message or error description.
				\item The number of calibration samples used for training.
			\end{itemize}
		\end{itemize}
	\end{itemize}
	
	\paragraph{Automatic save/load behavior.}
	
	\begin{itemize}[leftmargin=1.5em]
		\item Trained models are automatically saved by the SDK when \texttt{trainFocusModel} completes successfully.
		\item When performing inference, the SDK automatically loads the corresponding model based on \texttt{subjectId}; no explicit save/load API is required from the client.
	\end{itemize}
	
	\subsection*{2.3 Focus Inference}
	
	After a model has been trained for a subject, the SDK can perform real-time inference on live data coming from the connected device and output a focus/attention score.
	
	\paragraph{Start focus inference (will be provided).}
	
	\begin{itemize}[leftmargin=1.5em]
		\item \textbf{Function:} \texttt{startFocusInference(subjectId, updateHz, callback)}
		\begin{itemize}
			\item Loads the trained model associated with the specified subject ID.
			\item Uses live EEG from the connected device as testing data.
			\item At the specified update frequency (\texttt{updateHz}), the SDK:
			\begin{itemize}
				\item Reads an internal window of EEG data.
				\item Applies the subject-specific model.
				\item Calls the callback with:
				\begin{itemize}
					\item Focus/attention score (e.g., 1--100).
					\item Timestamp.
					\item Optional RSSI.
					\item Optional wearing status.
				\end{itemize}
			\end{itemize}
			\item If no trained model exists for the subject, an appropriate error is reported.
		\end{itemize}
	\end{itemize}
	
	\paragraph{Stop focus inference (will be provided).}
	
	\begin{itemize}[leftmargin=1.5em]
		\item \textbf{Function:} \texttt{stopFocusInference()}
		\begin{itemize}
			\item Stops performing focus inference on live data.
			\item The SDK ceases calling the focus inference callback.
		\end{itemize}
	\end{itemize}
	
	\section*{Task 3: Error Handling \& Logging}
	
	\subsection*{3.1 Error Reporting}
	
	\paragraph{Error callback (will be provided).}
	
	\begin{itemize}[leftmargin=1.5em]
		\item \textbf{Function:} \texttt{setErrorCallback(callback)}
		\begin{itemize}
			\item Registers a callback which receives:
			\begin{itemize}
				\item An error code.
				\item A short error message.
			\end{itemize}
			\item Used for reporting connection issues, calibration errors, model training errors, inference errors, etc.
		\end{itemize}
	\end{itemize}
	
	\subsection*{3.2 Debug Logging}
	
	\paragraph{Debug logging control (will be provided).}
	
	\begin{itemize}[leftmargin=1.5em]
		\item \textbf{Function:} \texttt{enableDebugLogging(enable)}
		\begin{itemize}
			\item Enables or disables internal SDK debug logging.
			\item Intended for development and troubleshooting.
		\end{itemize}
	\end{itemize}
	
	\section*{Demo Application Deliverables}
	
	A demo application will be delivered that demonstrates:
	
	\begin{itemize}[leftmargin=1.5em]
		\item Scanning and connecting to the iFocus device.
		\item Displaying RSSI as a link-quality indicator.
		\item Displaying wearing status using \texttt{getWearingStatus()}.
		\item Running calibration sessions:
		\begin{itemize}
			\item Start calibration (focus).
			\item Stop calibration.
			\item Start calibration (relax).
			\item Stop calibration.
		\end{itemize}
		\item Training the focus model for a subject using collected calibration data.
		\item Running focus inference and displaying the focus score over time.
		\item Showing errors and logs for debugging.
	\end{itemize}
	
	\section*{SDK Versioning, Firmware \& License}
	
	\begin{itemize}[leftmargin=1.5em]
		\item SDK documentation will specify:
		\begin{itemize}
			\item Supported Android versions and CPU architecture.
			\item Minimum required firmware version for the iFocus device.
			\item SDK versioning and compatibility notes.
		\end{itemize}
		\item License and activation requirements (if any) will be documented separately according to commercial agreements.
	\end{itemize}
	
	\section*{Final Function Checklist for Delivery}
	
	Total number of public SDK functions described in this document: \textbf{14}.
	
	\begin{itemize}[leftmargin=1.5em]
		\item $\square$ \texttt{startDeviceScan(callback)}
		\item $\square$ \texttt{stopDeviceScan()}
		\item $\square$ \texttt{connect(deviceId, autoReconnect, callback)}
		\item $\square$ \texttt{disconnect()}
		\item $\square$ \texttt{getCurrentRssi()}
		\item $\square$ \texttt{getWearingStatus()}
		\item $\square$ \texttt{startCalibration(subjectId, stateLabel)}
		\item $\square$ \texttt{stopCalibration()}
		\item $\square$ \texttt{trainFocusModel(subjectId, callback)}
		\item $\square$ \texttt{startFocusInference(subjectId, updateHz, callback)}
		\item $\square$ \texttt{stopFocusInference()}
		\item $\square$ \texttt{setErrorCallback(callback)}
		\item $\square$ \texttt{enableDebugLogging(enable)}
	\end{itemize}
	
\end{document}
